\documentclass{article} % this tells LaTeX to make an article (as opposed to a book, for example)

\usepackage{amsmath, % this adds functions for formatting equations nicely
			amssymb} % this gives us lots of greek symbols

\newtheorem{definition}{Definition} % this lets us create definition blocks as below
\newtheorem{remark}{Remark}

\title{Revealed Preference Theory}

\begin{document}	

\maketitle

\section{Setup and Definitions}

\textbf{Setup:}
Let $X=(x_{1}, x_{2}, \ldots, x_{n})$ and $Y=(y_{1}, y_{2}, \ldots, y_{n})$ be vectors representing bundles of goods chosen by the same individual at two different times. Each $x_{i}\in X$ and $y_{i}\in Y$ denote the quantity of good $i$ chosen in bundle X and Y, respectively. Further, let $P=(p_{1}, p_{2}, \ldots, p_{n})$ be the vector of available prices at the time when bundle $X$ was chosen such that $p_{i}\in P$ denotes the price of good $i$. Then, consider the following definitions:

\begin{definition}
We say that $X$ is \textbf{directly revealed preferred} to $Y$ if $P\cdot Y \leq P\cdot X$ (i.e. $p_{1}y_{1}+p_{2}y_{2}+\ldots+p_{n}y_{n}\leq p_{1}x_{1}+p_{2}x_{2}+\ldots+p_{n}x_{n}$).
\end{definition}

\begin{definition}
We say that $X$ is \textbf{strictly directly revealed preferred} to $Y$ if $P\cdot Y < P\cdot X$ (i.e. $p_{1}y_{1}+p_{2}y_{2}+\ldots+p_{n}y_{n}<p_{1}x_{1}+p_{2}x_{2}+\ldots+p_{n}x_{n}$).
\end{definition}

\begin{definition}
Consider three bundles, $X$, $Y$, and $Z$. Then $X$ is \textbf{indirectly revealed preferred} to $Z$ if $X$ is directly revealed preferred to $Y$ and $Y$ is directly revealed preferred to $Z$.
\end{definition}

\begin{definition}
We say $X$ is \textbf{revealed preferred} to $Y$ if $X$ is either directly revealed preferred or indirectly revealed preferred to $Y$.
\end{definition}

\begin{definition}
A set of choice data satisfies the \textbf{Weak Axiom of Revealed Preferences (WARP)} if for all bundles $X$ and $Y$, if $X$ is directly revealed preferred to $Y$, then $Y$ is not directly revealed preferred to $X$.
\end{definition}

\begin{remark}
If a consumer exhibits maximizing behavior, then their set of choice data will satisfy WARP. However, the converse isn't necessarily true when X and Y represent bundles from commodity spaces higher than two dimensions (i.e. there are more than two available goods and thus, $X$, $Y$, and $P$ are n-dimensional vectors such that $n\geq 2$). 
\end{remark}

\begin{definition}
A set of choice data satisfies the \textbf{Strong Axiom of Revealed Preferences (SARP)} if for all bundles $X$ and $Y$, if $X$ is revealed preferred (directly or indirectly) to $Y$, then $Y$ is not revealed preferred (directly or indirectly) to $X$.
\end{definition}

\begin{remark}
A consumer's set of choice data will satisfy SARP if and only if they exhibit maximizing behavior. 
\end{remark}

\begin{definition}
A set of choice data satisfies the \textbf{Weak Generalized Axiom of Revealed Preferences (WGARP)} if for all bundles $X$ and $Y$, if $X$ is directly revealed preferred to $Y$, then $Y$ is not strictly directly revealed preferred to $X$.
\end{definition}

\begin{definition}
A set of choice data satisfies the \textbf{Generalized Axiom of Revealed Preferences (GARP)} if for all bundles $X$ and $Y$, if $X$ is revealed preferred (directly or indirectly) to $Y$, then $Y$ is not strictly directly revealed preferred to $X$.
\end{definition}

\begin{remark}
The key difference between SARP and GARP requirements are that GARP allows consumers to have multiple bundles that maximizes utility under the same price levels (i.e. consumers are allowed to exhibit preferences that treat goods as perfect substitutes of each other).
\end{remark}

\section{Examples}

\end{document}